\documentclass{article}

%%% PACKAGES %%%
\usepackage[a4paper, margin=0.75in]{geometry}
\usepackage{setspace}
\usepackage{authblk}
\usepackage{graphicx}
\usepackage{blindtext}
\usepackage{multicol}
\usepackage{caption}
\usepackage{titlesec}
\usepackage{svg}
\usepackage{enumitem}

%%% CONFIG %%%
\renewcommand{\thesection}{\Alph{section}}
\renewcommand{\thesubsection}{\arabic{subsection}}
\titleformat{\section}{\centering\large\bfseries}{Part \thesection}{0em}{}{}
\titleformat{\subsection}{\bfseries}{\thesubsection.}{0.5em}{}{}
\newenvironment{mcfigure}
    {\par\noindent\minipage{\linewidth}}
    {\endminipage\par}
\setlist[enumerate]{
    label=(\alph*),
    wide=0pt,
}
\setlength{\abovecaptionskip}{5pt}
\setlength{\belowcaptionskip}{0pt}

%%% TITLE METADATA %%%
\title{CSC311 Final Report: Project Option 1}
\author[]{Leo Peckham}
\author[]{Longyue Wang}
\author[]{Gursewak Sandhu}
\affil[]{}
\date{}

\begin{document}
\maketitle

\begin{multicols}{2}

\section{}

\subsection{$k$-nearest neighbours}

\begin{enumerate}

\item
~
\vspace{-0.2cm}
\begin{mcfigure}
    \centering
    \includesvg[width=\linewidth]{A1a}
    {\footnotesize
        \setlength{\tabcolsep}{3pt}
        \begin{tabular}{ |c|c|c|c|c|c|c| } 
            \hline
            \textbf{$k$-value} & 1 & 6 & {\bf 11} & 16 & 21 & 26 \\ 
            \textbf{Accuracy} & 0.628 & 0.677 & {\bf 0.689}
                              & 0.675 & 0.668 & 0.651 \\ 
            \hline
        \end{tabular}
    }
    \captionof{figure}{Accuracies of user-clustering {\sc knn}}
    \label{fig:userclustering}
\end{mcfigure}


\item
From this data, we can see that the $k^*$ that gave us the best validation
accuracy was $k^* = 11$. Running on test we achieve an accuracy of $0.683$.

\item
Question-based clustering assumes that if two questions have very similar
distributions of answers across the known students, then the questions will
behave similarly for new students. Intuitively, if two questions ask about a
specific theorem, then the same students who get the first one wrong because
they forgot the statement of the theorem will get the second wrong as well.

\begin{mcfigure}
    \centering
    \includesvg[width=\linewidth]{A1c}
    {\footnotesize
        \setlength{\tabcolsep}{3pt}
        \begin{tabular}{ |c|c|c|c|c|c|c| } 
            \hline
            \textbf{$k$-value} & 1 & 6 & 11 & 16 & {\bf 21} & 26 \\ 
            \textbf{Accuracy} & 0.618 & 0.660 & 0.680
                              & 0.687 & {\bf 0.690} & 0.689 \\ 
            \hline
        \end{tabular}
    }
    \captionof{figure}{Accuracies of question-clustering KNN}
    \label{fig:questionclustering}
\end{mcfigure}

From this data, we can see that the $k^*$ that gave us the best validation
accuracy was $k^* = 21$. Running on test we achieve an accuracy of $0.670$.

\item
The performances across the board are similar. The user-based approach does
better on test, but only by around a tenth of a percent accuracy. The graphs
for different $k$-values do look quite different, though, with the
user-clustering in figure \ref{fig:userclustering} falling off much faster for
high $k$-values than the question-clustering in figure
\ref{fig:questionclustering} .

\item

This method relies on a sufficient amount of data to be able to accurately
cluster. If there is an example with a significant amount held-out, it will
become inaccurate because many nearby users/questions will look similar.

The dimension of the features is also a problem. First, it can be extremely
high, being either the number of students or the number of questions. This
poses a problem for a {\sc knn} approach, especially one that uses a Euclidean
distance like this one. Mapping to a smaller latent space first and using
another distance function would help alieviate this problem. Second, if we add
new students or questions to the dataset, the very {\it dimensionality} of our
data will change. This makes it more difficult to improve our model with new
data; mapping to a laten space would also fix this issue.

\end{enumerate}

\subsection{Item response theory}
\blindtext

\subsection{Matrix factorization}
\blindtext

\subsection{Ensemble}
\blindtext

\section{}

\blindtext

\end{multicols}
\end{document}
